\chapter{Documentação}
O Analisador Sintático tem a função de solicitar ao Analisador léxico o próximo Tokens desde o inicio do programa ate o final, e apos a cada solicitação, verificar se o Token lido esta em consonância com o que foi determinado pela gramatica do programa.
\newline
O Projeto foi desenvolvido em conjunto com as plataformas Visual Studio Code e Apache Netbeans, a principio todo o código seria criado somente "a mão" criando as pastas e arquivos manualmente, porém, devido a algumas facilidades que o Netbeans  oferece, o projeto foi alterado em alguns parâmetros para que seja possível abrir no Netbeans.
Uma dessas facilidades é a opção de geração da Documentação do projeto, assim o fizemos e esta documentação gerada automaticamente pelo Netbeans se encontra no diretório: " /TP{\_}Compilador/Projeto/dist/javadoc/" . Para esta implementação foi acatado o modelo abordado pelo autor do livro base da disciplina: Aho .
\section{Tabela Frist-Follow e LL(1)}

Para evitar o retrocesso métodos baseados em 
tabela são utilizados. Para auxiliar na construção 
das tabelas sintáticas deve serutilizado as funções first(α)
e follow(A)
\subsection{Frist}

Defina First(α), onde α é qualquer cadeia de 
símbolos da gramática, como sendo o conjunto 
de símbolos terminais que iniciam as cadeias 
derivadas de α. Se α → * ε, então ε está no 
First(α)
\subsection{Calculando Frist(X)}

Para calcular o First(X) para todo símbolo X da 
gramática, aplique as seguintes regras até que não haja 
mais terminais ou e que possam ser acrescentados a 
algum dos conjuntos First.


\begin{itemize}
    \item 
       Se X é um símbolo terminal, então First(X)= {X}
    \item
        Se X é um símbolo não-terminal e X → Y
1
Y
2
..Y
k
 é 
uma produção para k ≥ 1, então acrescente a ao 
First(X) se, para algum i, a  estiver em First( Y\i ), e ε
estiver em todos os First(Y\1),...,First(Y\i-1); ou seja,Y\1
...Y\i-1 → ε. Se ε está em First(Y\ j ) para todo j=1, 2, …, k então ε está em First(X).

    \item
       Se X → e é uma produção de X então ε está em  First(X).
    
   \end{itemize}
\section{Follow(A)}

Defina Follow(A), para o não-termianl A, como 
sendo o conjunto de terminais a que podem 
aparecer imediatamente à direita de A em uma 
forma sentencial; ou seja, o conjunto de terminais 
a tais que exista uma derivação na forma S → 
αAaβ, para algum α, β.
\subsection{Calculando Follow(A)}

Para calcular o Follow(A) para todos os nãoterminais da gramática aplique as seguintes 
regras até que nada mais possa ser 
acresncentado a nenhum dos conjuntos Follow

   
\begin{itemize}
    
       \item  Coloque {\$} (símbolo de final de cadeia) em 
Follow(S), onde S é o símbolo inicial da 
gramática.
\item
 Se houver uma produção A → αBβ, então tudo 
no First(β) exceto ε está em Follow(B).
\item
Se houver uma produção A → αB, ou uma 
produção A → αBβ onde Firts(β) contém ε , 
então inclua Follow(A) ao Follow(B).
   \end{itemize}
   
   
\section{LL(1)}

Análise LL(1)
   
\begin{itemize}
\item Conceitualmente, o analisador LL(1) constrói uma derivação mais à esquerda
para o programa, partindo do símbolo inicial
\item A cada passo da derivação, o prefixo de terminais da forma sentencial tem 
que casar com um prefixo da entrada
\item Caso exista mais de uma regra para o não-terminal que vai gerar o próximo 
passo da derivação, o analisador usa o primeiro token após esse prefixo para 
escolher qual regra usar

\item Esse processo continua até todo o programa ser derivado ou acontecer um 
erro (o prefixo de terminais da forma sentencial não casa com um prefixo do 
programa)

   \end{itemize}
   Assim, foi criado as tabelas de Frist e Follow da Gramatica e estao dispostas no arquivo Criado pelo Programa Microsoft Excel.
   O Arquivo encontra-se na pasta raiz deste trabalho,pode ser encontrado pelo nome : "AnaliseSintatica\_FristFollow\_LL1"
   
   