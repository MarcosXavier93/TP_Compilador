\chapter{Testes realizados}
Para testar todo o compilador construído, foram propostos sete exemplos de teste, abordando algumas possibilidade de programas a serem compilados.
Apartir desses arquivos podemos visualizar os seguintes retorno do compilador :
\newline

\section{Teste 1}

\begin{lstlisting}[caption={Teste1.txt},label={lst:label},language=C]

init
a, b, $c, is integer;
result is real;
write("Digite o valor de a:"
read (a);
write("Digite o valor de c:"
read (c);
b := 10
result := (a * c)/(b + 5 - 345);
write("O resultado e:");
write(result);
stop


\end{lstlisting}\newline
Para o código acima  o compilador obteve a seguinte saida :

\begin{lstlisting}[caption={Saida para o Codigo de teste  : Teste1.txt},label={Entrada 1},language=C]
**** Inicio Parser ****
Token Consumido(1): < init_program >
Token Consumido(2): < identifier, a >
Token Consumido(2): < virgula >
Token Consumido(2): < identifier, b >
Token Consumido(2): < virgula >
Token Consumido(2): < identifier, c >
Token Consumido(2): < virgula >
Error in: identList; Error(2): Token n?£o esperado:< is_decl >
Fim de arquivo inesperado.


\end{lstlisting}\newline
NO exemplo acima, a declaração da ultima variável foi declarada errada, dessa forma, o programa é abortado.

\subsection{Correção do Teste 1}
Foram feitas as devidas correções no código acima para adequar as regras da gramatica da linguagem .Assim, obtemos o seguinte resultado.


\begin{lstlisting}[caption={Saida Correta para o Codigo de teste  : Teste1.txt},label={Entrada 1},language=C]
**** Tokens lidos ****
< init_program >
< identifier, a >
< virgula >      
< identifier, b >
< virgula >
Error  na Linha(2): Token '$' Nao esperado
< identifier, c >
< is_decl >
< integer_type >
< ponto_virgula >
< identifier, result >
< is_decl >
< real_type >
< ponto_virgula >
< begin >
< write >
< abre_parent >
< literal, "Digite o valor de a:" >
< fecha_parent >
< ponto_virgula >
< read >
< abre_parent >
< identifier, a >
< fecha_parent >
< ponto_virgula >
< write >
< abre_parent >
< literal, "Digite o valor de c:" >
< fecha_parent >
< ponto_virgula >
< read >
< abre_parent >
< identifier, c >
< fecha_parent >
< ponto_virgula >
< identifier, b >
< assign >
< num, 10 >
< ponto_virgula >
< identifier, result >
< assign >
< abre_parent >
< identifier, a >
< mult_mulop >
< identifier, c >
< fecha_parent >
< div_mulop >
< abre_parent >
< identifier, b >
< soma_addop >
< num, 5 >
< menos_addop >
< num, 345 >
< fecha_parent >
< ponto_virgula >
< write >
< abre_parent >
< literal, "O resultado e:" >
< fecha_parent >
< ponto_virgula >
< write >
< abre_parent >
< identifier, result >
< fecha_parent >
< ponto_virgula >
< stop_program >



**** Inicio Parser ****
Token Consumido(1): < init_program >
Token Consumido(2): < identifier, a >
Token Consumido(2): < virgula >
Token Consumido(2): < identifier, b >
Token Consumido(2): < virgula >
Token Consumido(2): < identifier, c >
Token Consumido(2): < is_decl >
Token Consumido(2): < integer_type >
Token Consumido(2): < ponto_virgula >
Token Consumido(3): < identifier, result >
Token Consumido(3): < is_decl >
Token Consumido(3): < real_type >
Token Consumido(3): < ponto_virgula >
Token Consumido(4): < begin >
Token Consumido(5): < write >
Token Consumido(5): < abre_parent >
Token Consumido(5): < literal, "Digite o valor de a:" >
Token Consumido(5): < fecha_parent >
Token Consumido(5): < ponto_virgula >
Token Consumido(6): < read >
Token Consumido(6): < abre_parent >
Token Consumido(6): < identifier, a >
Token Consumido(6): < fecha_parent >
Token Consumido(6): < ponto_virgula >
Token Consumido(7): < write >
Token Consumido(7): < abre_parent >
Token Consumido(7): < literal, "Digite o valor de c:" >
Token Consumido(7): < fecha_parent >
Token Consumido(7): < ponto_virgula >
Token Consumido(8): < read >
Token Consumido(8): < abre_parent >
Token Consumido(8): < identifier, c >
Token Consumido(8): < fecha_parent >
Token Consumido(8): < ponto_virgula >
Token Consumido(9): < identifier, b >
Token Consumido(9): < assign >
Token Consumido(9): < num, 10 >
Token Consumido(9): < ponto_virgula >
Token Consumido(10): < identifier, result >
Token Consumido(10): < assign >
Token Consumido(10): < abre_parent >
Token Consumido(10): < identifier, a >
Token Consumido(10): < mult_mulop >
Token Consumido(10): < identifier, c >
Token Consumido(10): < fecha_parent >
Token Consumido(10): < div_mulop >
Token Consumido(10): < abre_parent >
Token Consumido(10): < identifier, b >
Token Consumido(10): < soma_addop >
Token Consumido(10): < num, 5 >
Token Consumido(10): < menos_addop >
Token Consumido(10): < num, 345 >
Token Consumido(10): < fecha_parent >
Token Consumido(10): < ponto_virgula >
Token Consumido(11): < write >
Token Consumido(11): < abre_parent >
Token Consumido(11): < literal, "O resultado e:" >
Token Consumido(11): < fecha_parent >
Token Consumido(11): < ponto_virgula >
Token Consumido(12): < write >
Token Consumido(12): < abre_parent >
Token Consumido(12): < identifier, result >
Token Consumido(12): < fecha_parent >
Token Consumido(12): < ponto_virgula >



\end{lstlisting}\newline




% Teste 2 -----------------------------------------------------
\section{Teste 2}
\begin{lstlisting}[caption={Teste2.txt},label={lst:label},language=C]a, a, _valor, b_1, b_2 : integer;
init
write("Entre com o valor de a:");
read (a);
b_1 := a * a;
write("O valor de b1 e:");
write (b_1);
b_2 = b + a/2 * (a + 5);
write("O valor de b1 e:");
Write (b1);
stop

\end{lstlisting}\newline

Para o código acima o compilador obteve a seguinte saida :

\begin{lstlisting}[caption={Saida para o Codigo de teste  : Teste2.txt},label={Entrada 1},language=Correta C]
 
**** Tokens lidos ****
< identifier, a >
< virgula >
Error  na Linha(1): Token '_' Nao esperado
< identifier, b_1 >
< virgula >
< identifier, b_2 >
Error  na Linha(1): Token ':' Nao esperado
< integer_type >
< ponto_virgula >
< init_program >
< write >
< abre_parent >
< literal, "Entre com o valor de a:" >
< fecha_parent >
< ponto_virgula >
< read >
< abre_parent >
< identifier, a >
< fecha_parent >
< ponto_virgula >
< identifier, b_1 >
< assign >
< identifier, a >
< mult_mulop >
< identifier, a >
< ponto_virgula >
< write >
< abre_parent >
< literal, "O valor de b1 e:" >
< fecha_parent >
< ponto_virgula >
< write >
< abre_parent >
< identifier, b_1 >
< fecha_parent >
< ponto_virgula >
< identifier, b_2 >
< equal_relop >
< identifier, b >
< soma_addop >
< identifier, a >
< div_mulop >
< num, 2 >
< mult_mulop >
< abre_parent >
< identifier, a >
< soma_addop >
< num, 5 >
< fecha_parent >
< ponto_virgula >
< write >
< abre_parent >
< literal, "O valor de b1 e:" >
< fecha_parent >
< ponto_virgula >
< write >
< abre_parent >
< identifier, b1 >
< fecha_parent >
< ponto_virgula >
< stop_program >



**** Inicio Parser ****
Error in: program; Error(1): Token n?£o esperado:< identifier, a >
\end{lstlisting}\newline
No Exemplo acima, foi reportado um Token não esperado, pois se esperava o identificador init para inicio de qualquer programa.


\subsection{Correção do Teste 2}
Foram feitas as devidas correções no código acima para adequar as regras da gramatica da linguagem .Assim, obtemos o seguinte resultado.


\begin{lstlisting}[caption={Saida Correta para o Codigo de teste  : Teste2.txt},label={Entrada 1},language=C]

\end{lstlisting}\newline

%Teste 3 --------------------------------------

\section{Teste 3}


\newline
\begin{lstlisting}[caption={Teste3.txt},label={lst:label}]
% Programa de Teste Calculo de idade %
INIT
cont is int;
media, idade, soma is integer;
begin
cont := 5;
soma := 0;
do
write("Altura:" );
read (altura);
soma := soma+altura;
cont := cont - 1;
while(cont > 0)
media := soma / qts
write("Media: ");
write (media);
STOP
\end{lstlisting}


Para o código acima o compilador obteve a seguinte saída :

\begin{lstlisting}[caption={Saída para o Código de teste  : Teste3.txt},label={Entrada 1},language=C]
**** Tokens lidos ****
< init_program >
< identifier, cont >
< is_decl >
< identifier, int >
< ponto_virgula >
< identifier, media >
< virgula >
< identifier, idade >
< virgula >
< identifier, soma >
< is_decl >
< integer_type >
< ponto_virgula >
< begin >
< identifier, cont >
< assign >
< num, 5 >
< ponto_virgula >
< identifier, soma >
< assign >
< num, 0 >
< ponto_virgula >
< do >
< write >
< abre_parent >
< literal, "Altura:" >
< fecha_parent >
< ponto_virgula >
< read >
< abre_parent >
< identifier, altura >
< fecha_parent >
< ponto_virgula >
< identifier, soma >
< assign >
< identifier, soma >
< soma_addop >
< identifier, altura >
< ponto_virgula >
< identifier, cont >
< assign >
< identifier, cont >
< menos_addop >
< num, 1 >
< ponto_virgula >
< while >
< abre_parent >
< identifier, cont >
< greater_than_relop >
< num, 0 >
< fecha_parent >
< identifier, media >
< assign >
< identifier, soma >
< div_mulop >
< identifier, qts >
< write >
< abre_parent >
< literal, "Media: " >
< fecha_parent >
< ponto_virgula >
< write >
< abre_parent >
< identifier, media >
< fecha_parent >
< ponto_virgula >
< stop_program >



**** Inicio Parser ****
Token Consumido(3): < init_program >
Token Consumido(4): < identifier, cont >
Token Consumido(4): < is_decl >
Error in: type; Error(4): Token n?£o esperado:< identifier, int >
Fim de arquivo inesperado.
\end{lstlisting}
No exemplo acima , o programa reportou um erro pois foi definido um tipo de variável inexistente na gramatica para essa linguagem. O tipo de variável int não foi modelado para essa gramatica.

\subsection{Correção do Teste 3}
Foram feitas as devidas correções no código acima para adequar as regras da gramatica da linguagem .Assim, obtemos o seguinte resultado.


\begin{lstlisting}[caption={Saida Correta para o Codigo de teste  : Teste3.txt},label={Entrada 1},language=C]
**** Tokens lidos ****
< init_program >
< identifier, cont >
< virgula >
< identifier, qts >
< is_decl >
< integer_type >
< ponto_virgula >
< identifier, media >
< virgula >
< identifier, idade >
< virgula >
< identifier, soma >
< is_decl >
< integer_type >
< ponto_virgula >
< begin >
< identifier, cont >
< assign >
< num, 5 >
< ponto_virgula >
< identifier, soma >
< assign >
< num, 0 >
< ponto_virgula >
< identifier, qts >
< assign >
< num, 1 >
< ponto_virgula >
< do >
< write >
< abre_parent >
< literal, "Altura:" >
< fecha_parent >
< ponto_virgula >
< read >
< abre_parent >
< identifier, altura >
< fecha_parent >
< ponto_virgula >
< identifier, soma >
< assign >
< identifier, soma >
< soma_addop >
< identifier, altura >
< ponto_virgula >
< identifier, cont >
< assign >
< identifier, cont >
< menos_addop >
< num, 1 >
< ponto_virgula >
< while >
< abre_parent >
< identifier, cont >
< greater_than_relop >
< num, 0 >
< fecha_parent >
< ponto_virgula >
< identifier, media >
< assign >
< identifier, soma >
< div_mulop >
< identifier, qts >
< ponto_virgula >
< write >
< abre_parent >
< literal, "Media: " >
< fecha_parent >
< ponto_virgula >
< write >
< abre_parent >
< identifier, media >
< fecha_parent >
< ponto_virgula >
< stop_program >



**** Inicio Parser ****
Token Consumido(3): < init_program >
Token Consumido(4): < identifier, cont >
Token Consumido(4): < virgula >
Token Consumido(4): < identifier, qts >
Token Consumido(4): < is_decl >
Token Consumido(4): < integer_type >
Token Consumido(4): < ponto_virgula >
Token Consumido(5): < identifier, media >
Token Consumido(5): < virgula >
Token Consumido(5): < identifier, idade >
Token Consumido(5): < virgula >
Token Consumido(5): < identifier, soma >
Token Consumido(5): < is_decl >
Token Consumido(5): < integer_type >
Token Consumido(5): < ponto_virgula >
Token Consumido(6): < begin >
Token Consumido(7): < identifier, cont >
Token Consumido(7): < assign >
Token Consumido(7): < num, 5 >
Token Consumido(7): < ponto_virgula >
Token Consumido(8): < identifier, soma >
Token Consumido(8): < assign >
Token Consumido(8): < num, 0 >
Token Consumido(8): < ponto_virgula >
Token Consumido(9): < identifier, qts >
Token Consumido(9): < assign >
Token Consumido(9): < num, 1 >
Token Consumido(9): < ponto_virgula >
Token Consumido(10): < do >
Token Consumido(11): < write >
Token Consumido(11): < abre_parent >
Token Consumido(11): < literal, "Altura:" >
Token Consumido(11): < fecha_parent >
Token Consumido(11): < ponto_virgula >
Token Consumido(12): < read >
Token Consumido(12): < abre_parent >
Token Consumido(12): < identifier, altura >
Token Consumido(12): < fecha_parent >
Token Consumido(12): < ponto_virgula >
Token Consumido(13): < identifier, soma >
Token Consumido(13): < assign >
Token Consumido(13): < identifier, soma >
Token Consumido(13): < soma_addop >
Token Consumido(13): < identifier, altura >
Token Consumido(13): < ponto_virgula >
Token Consumido(14): < identifier, cont >
Token Consumido(14): < assign >
Token Consumido(14): < identifier, cont >
Token Consumido(14): < menos_addop >
Token Consumido(14): < num, 1 >
Token Consumido(14): < ponto_virgula >
Token Consumido(15): < while >
Token Consumido(15): < abre_parent >
Token Consumido(15): < identifier, cont >
Token Consumido(15): < greater_than_relop >
Token Consumido(15): < num, 0 >
Token Consumido(15): < fecha_parent >
Token Consumido(15): < ponto_virgula >
Token Consumido(16): < identifier, media >
Token Consumido(16): < assign >
Token Consumido(16): < identifier, soma >
Token Consumido(16): < div_mulop >
Token Consumido(16): < identifier, qts >
Token Consumido(16): < ponto_virgula >
Token Consumido(17): < write >
Token Consumido(17): < abre_parent >
Token Consumido(17): < literal, "Media: " >
Token Consumido(17): < fecha_parent >
Token Consumido(17): < ponto_virgula >
Token Consumido(18): < write >
Token Consumido(18): < abre_parent >
Token Consumido(18): < identifier, media >
Token Consumido(18): < fecha_parent >
Token Consumido(18): < ponto_virgula >
\end{lstlisting}\newline




%-------------------Teste 4 ------------------------
\section{Teste 4}


\begin{lstlisting}[caption={Teste4.txt},label={lst:label},language=C]
init
% Outro programa de teste
i, j, k, @total is integer;
nome is string
write("Digite o seu nome: ");
read(nome);
write("Digite um valor inteiro: );
read (I);
k := i * (5-i * 50 / 10;
j := i * 10;
k := i* j / k;
k := 4 + a $;
write(nome);
write(", os números gerados sao: ");
write(i);
write(j);
write(k);
\end{lstlisting}

Para o codigo acima o compilador obteve a seguinte saida :

\begin{lstlisting}[caption={Saida para o Codigo de teste  : Teste4.txt},label={Entrada 1},language=C]
**** Tokens lidos ****
< init_program >
Error(2): coment?¡rio n?£o fechado



**** Inicio Parser ****
Token Consumido(1): < init_program >
Fim de arquivo inesperado.


\end{lstlisting}
No código acima, não foi fechado o comentário, assim, erro para iniciar o programa, pois era esperado o {\%} para fim de comentário.


\subsection{Correção do Teste 4}
Foram feitas as devidas correções no código acima para adequar as regras da gramatica da linguagem .Assim, obtemos o seguinte resultado.


\begin{lstlisting}[caption={Saida Correta para o Codigo de teste  : Teste4.txt},label={Entrada 1},language=C]


**** Tokens lidos ****
< init_program >
< identifier, i >
< virgula >
< identifier, j >
< virgula >
< identifier, k >
< virgula >
Error  na Linha(3): Token '@' Nao esperado
< identifier, total >
< is_decl >
< integer_type >
< ponto_virgula >
< identifier, nome >
< is_decl >
< string_type >
< ponto_virgula >
< begin >
< write >
< abre_parent >
< literal, "Digite o seu nome: " >
< fecha_parent >
< ponto_virgula >
< read >
< abre_parent >
< identifier, nome >
< fecha_parent >
< ponto_virgula >
< write >
< abre_parent >
< literal, "Digite um valor inteiro: " >
< fecha_parent >
< ponto_virgula >
< read >
< abre_parent >
< identifier, i >
< fecha_parent >
< ponto_virgula >
< identifier, k >
< assign >
< abre_parent >
< abre_parent >
< identifier, i >
< mult_mulop >
< num, 5 >
< fecha_parent >
< menos_addop >
< abre_parent >
< identifier, i >
< mult_mulop >
< num, 50 >
< fecha_parent >
< fecha_parent >
< div_mulop >
< num, 10 >
< ponto_virgula >
< identifier, j >
< assign >
< identifier, i >
< mult_mulop >
< num, 10 >
< ponto_virgula >
< identifier, k >
< assign >
< abre_parent >
< identifier, i >
< mult_mulop >
< identifier, j >
< fecha_parent >
< div_mulop >
< identifier, k >
< ponto_virgula >
< identifier, k >
< assign >
< num, 4 >
< soma_addop >
< identifier, a >
Error  na Linha(13): Token '$' Nao esperado
< ponto_virgula >
< write >
< abre_parent >
< identifier, nome >
< fecha_parent >
< ponto_virgula >
< write >
< abre_parent >
< literal, ", os n?ºmeros gerados sao: " >
< fecha_parent >
< ponto_virgula >
< write >
< abre_parent >
< identifier, i >
< fecha_parent >
< ponto_virgula >
< write >
< abre_parent >
< identifier, j >
< fecha_parent >
< ponto_virgula >
< write >
< abre_parent >
< identifier, k >
< fecha_parent >
< ponto_virgula >
< stop_program >



**** Inicio Parser ****
Token Consumido(1): < init_program >
Token Consumido(3): < identifier, i >
Token Consumido(3): < virgula >
Token Consumido(3): < identifier, j >
Token Consumido(3): < virgula >
Token Consumido(3): < identifier, k >
Token Consumido(3): < virgula >
Token Consumido(3): < identifier, total >
Token Consumido(3): < is_decl >
Token Consumido(3): < integer_type >
Token Consumido(3): < ponto_virgula >
Token Consumido(4): < identifier, nome >
Token Consumido(4): < is_decl >
Token Consumido(4): < string_type >
Token Consumido(4): < ponto_virgula >
Token Consumido(5): < begin >
Token Consumido(6): < write >
Token Consumido(6): < abre_parent >
Token Consumido(6): < literal, "Digite o seu nome: " >
Token Consumido(6): < fecha_parent >
Token Consumido(6): < ponto_virgula >
Token Consumido(7): < read >
Token Consumido(7): < abre_parent >
Token Consumido(7): < identifier, nome >
Token Consumido(7): < fecha_parent >
Token Consumido(7): < ponto_virgula >
Token Consumido(8): < write >
Token Consumido(8): < abre_parent >
Token Consumido(8): < literal, "Digite um valor inteiro: " >
Token Consumido(8): < fecha_parent >
Token Consumido(8): < ponto_virgula >
Token Consumido(9): < read >
Token Consumido(9): < abre_parent >
Token Consumido(9): < identifier, i >
Token Consumido(9): < fecha_parent >
Token Consumido(9): < ponto_virgula >
Token Consumido(10): < identifier, k >
Token Consumido(10): < assign >
Token Consumido(10): < abre_parent >
Token Consumido(10): < abre_parent >
Token Consumido(10): < identifier, i >
Token Consumido(10): < mult_mulop >
Token Consumido(10): < num, 5 >
Token Consumido(10): < fecha_parent >
Token Consumido(10): < menos_addop >
Token Consumido(10): < abre_parent >
Token Consumido(10): < identifier, i >
Token Consumido(10): < mult_mulop >
Token Consumido(10): < num, 50 >
Token Consumido(10): < fecha_parent >
Token Consumido(10): < fecha_parent >
Token Consumido(10): < div_mulop >
Token Consumido(10): < num, 10 >
Token Consumido(10): < ponto_virgula >
Token Consumido(11): < identifier, j >
Token Consumido(11): < assign >
Token Consumido(11): < identifier, i >
Token Consumido(11): < mult_mulop >
Token Consumido(11): < num, 10 >
Token Consumido(11): < ponto_virgula >
Token Consumido(12): < identifier, k >
Token Consumido(12): < assign >
Token Consumido(12): < abre_parent >
Token Consumido(12): < identifier, i >
Token Consumido(12): < mult_mulop >
Token Consumido(12): < identifier, j >
Token Consumido(12): < fecha_parent >
Token Consumido(12): < div_mulop >
Token Consumido(12): < identifier, k >
Token Consumido(12): < ponto_virgula >
Token Consumido(13): < identifier, k >
Token Consumido(13): < assign >
Token Consumido(13): < num, 4 >
Token Consumido(13): < soma_addop >
Token Consumido(13): < identifier, a >
Token Consumido(13): < ponto_virgula >
Token Consumido(14): < write >
Token Consumido(14): < abre_parent >
Token Consumido(14): < identifier, nome >
Token Consumido(14): < fecha_parent >
Token Consumido(14): < ponto_virgula >
Token Consumido(15): < write >
Token Consumido(15): < abre_parent >
Token Consumido(15): < literal, ", os n?ºmeros gerados sao: " >
Token Consumido(15): < fecha_parent >
Token Consumido(15): < ponto_virgula >
Token Consumido(16): < write >
Token Consumido(16): < abre_parent >
Token Consumido(16): < identifier, i >
Token Consumido(16): < fecha_parent >
Token Consumido(16): < ponto_virgula >
Token Consumido(17): < write >
Token Consumido(17): < abre_parent >
Token Consumido(17): < identifier, j >
Token Consumido(17): < fecha_parent >
Token Consumido(17): < ponto_virgula >
Token Consumido(18): < write >
Token Consumido(18): < abre_parent >
Token Consumido(18): < identifier, k >
Token Consumido(18): < fecha_parent >
Token Consumido(18): < ponto_virgula >


\end{lstlisting}\newline




%-------------------Teste 5 ------------------------
\section{Teste 5}



\begin{lstlisting}[caption={Teste5.txt},label={lst:label},language=C]
init
i, j, k, @total is integer;
nome is string
write("Digite o seu nome: ");
read(nome);
write("Digite um valor inteiro: );
read (I);
k := i * (5-i * 50 / 10;
j := i * 10;
k := i* j / k;
k := 4 + a $;
write(nome);
write(", os números gerados sao: ");
write(i);
write(j);
write(k);

\end{lstlisting}

Para o codigo acima o compilador obteve a seguinte saida :

\begin{lstlisting}[caption={Saida para o Codigo de teste  : Teste5.txt},label={Entrada 1},language=C]

**** Tokens lidos ****
< init_program >
< identifier, i >
< virgula >
< identifier, j >
< virgula >
< identifier, k >
< virgula >
Error  na Linha(2): Token '@' Nao esperado
< identifier, total >
< is_decl >
< integer_type >
< ponto_virgula >
< identifier, nome >
< is_decl >
< string_type >
< write >
< abre_parent >
< literal, "Digite o seu nome: " >
< fecha_parent >
< ponto_virgula >
< read >
< abre_parent >
< identifier, nome >
< fecha_parent >
< ponto_virgula >
< write >
< abre_parent >
Error  na Linha(6): Token '"' Nao esperado
< read >
< abre_parent >
< identifier, i >
< fecha_parent >
< ponto_virgula >
< identifier, k >
< assign >
< identifier, i >
< mult_mulop >
< abre_parent >
< num, 5 >
< menos_addop >
< identifier, i >
< mult_mulop >
< num, 50 >
< div_mulop >
< num, 10 >
< ponto_virgula >
< identifier, j >
< assign >
< identifier, i >
< mult_mulop >
< num, 10 >
< ponto_virgula >
< identifier, k >
< assign >
< identifier, i >
< mult_mulop >
< identifier, j >
< div_mulop >
< identifier, k >
< ponto_virgula >
< identifier, k >
< assign >
< num, 4 >
< soma_addop >
< identifier, a >
Error  na Linha(10): Token '$' Nao esperado
< ponto_virgula >
< write >
< abre_parent >
< identifier, nome >
< fecha_parent >
< ponto_virgula >
< write >
< abre_parent >
< literal, ", os n?ºmeros gerados sao: " >
< fecha_parent >
< ponto_virgula >
< write >
< abre_parent >
< identifier, i >
< fecha_parent >
< ponto_virgula >
< write >
< abre_parent >
< identifier, j >
< fecha_parent >
< ponto_virgula >
< write >
< abre_parent >
< identifier, k >
< fecha_parent >
< ponto_virgula >



**** Inicio Parser ****
Token Consumido(1): < init_program >
Token Consumido(2): < identifier, i >
Token Consumido(2): < virgula >
Token Consumido(2): < identifier, j >
Token Consumido(2): < virgula >
Token Consumido(2): < identifier, k >
Token Consumido(2): < virgula >
Token Consumido(2): < identifier, total >
Token Consumido(2): < is_decl >
Token Consumido(2): < integer_type >
Token Consumido(2): < ponto_virgula >
Token Consumido(3): < identifier, nome >
Token Consumido(3): < is_decl >
Token Consumido(3): < string_type >
Error in: declList; Error(4): Token n?£o esperado:< write >
Fim de arquivo inesperado.

\end{lstlisting}
No Código acima, não foi declarado o begin. Assim o comando write foi dado como inesperado, pois era esperado o identificador begin.

\subsection{Correção do Teste 5}
Foram feitas as devidas correções no código acima para adequar as regras da gramatica da linguagem .Assim, obtemos o seguinte resultado.


\begin{lstlisting}[caption={Saida Correta para o Codigo de teste  : Teste5.txt},label={Entrada 1},language=C]


**** Tokens lidos ****
< init_program >
< identifier, i >
< virgula >
< identifier, j >
< virgula >
< identifier, k >
< virgula >
Error  na Linha(3): Token '@' Nao esperado
< identifier, total >
< is_decl >
< integer_type >
< ponto_virgula >
< identifier, nome >
< is_decl >
< string_type >
< ponto_virgula >
< begin >
< write >
< abre_parent >
< literal, "Digite o seu nome: " >
< fecha_parent >
< ponto_virgula >
< read >
< abre_parent >
< identifier, nome >
< fecha_parent >
< ponto_virgula >
< write >
< abre_parent >
< literal, "Digite um valor inteiro: " >
< fecha_parent >
< ponto_virgula >
< read >
< abre_parent >
< identifier, i >
< fecha_parent >
< ponto_virgula >
< identifier, k >
< assign >
< abre_parent >
< abre_parent >
< identifier, i >
< mult_mulop >
< num, 5 >
< fecha_parent >
< menos_addop >
< abre_parent >
< identifier, i >
< mult_mulop >
< num, 50 >
< fecha_parent >
< fecha_parent >
< div_mulop >
< num, 10 >
< ponto_virgula >
< identifier, j >
< assign >
< identifier, i >
< mult_mulop >
< num, 10 >
< ponto_virgula >
< identifier, k >
< assign >
< abre_parent >
< identifier, i >
< mult_mulop >
< identifier, j >
< fecha_parent >
< div_mulop >
< identifier, k >
< ponto_virgula >
< identifier, k >
< assign >
< num, 4 >
< soma_addop >
< identifier, a >
Error  na Linha(13): Token '$' Nao esperado
< ponto_virgula >
< write >
< abre_parent >
< identifier, nome >
< fecha_parent >
< ponto_virgula >
< write >
< abre_parent >
< literal, ", os n?ºmeros gerados sao: " >
< fecha_parent >
< ponto_virgula >
< write >
< abre_parent >
< identifier, i >
< fecha_parent >
< ponto_virgula >
< write >
< abre_parent >
< identifier, j >
< fecha_parent >
< ponto_virgula >
< write >
< abre_parent >
< identifier, k >
< fecha_parent >
< ponto_virgula >
< stop_program >



**** Inicio Parser ****
Token Consumido(1): < init_program >
Token Consumido(3): < identifier, i >
Token Consumido(3): < virgula >
Token Consumido(3): < identifier, j >
Token Consumido(3): < virgula >
Token Consumido(3): < identifier, k >
Token Consumido(3): < virgula >
Token Consumido(3): < identifier, total >
Token Consumido(3): < is_decl >
Token Consumido(3): < integer_type >
Token Consumido(3): < ponto_virgula >
Token Consumido(4): < identifier, nome >
Token Consumido(4): < is_decl >
Token Consumido(4): < string_type >
Token Consumido(4): < ponto_virgula >
Token Consumido(5): < begin >
Token Consumido(6): < write >
Token Consumido(6): < abre_parent >
Token Consumido(6): < literal, "Digite o seu nome: " >
Token Consumido(6): < fecha_parent >
Token Consumido(6): < ponto_virgula >
Token Consumido(7): < read >
Token Consumido(7): < abre_parent >
Token Consumido(7): < identifier, nome >
Token Consumido(7): < fecha_parent >
Token Consumido(7): < ponto_virgula >
Token Consumido(8): < write >
Token Consumido(8): < abre_parent >
Token Consumido(8): < literal, "Digite um valor inteiro: " >
Token Consumido(8): < fecha_parent >
Token Consumido(8): < ponto_virgula >
Token Consumido(9): < read >
Token Consumido(9): < abre_parent >
Token Consumido(9): < identifier, i >
Token Consumido(9): < fecha_parent >
Token Consumido(9): < ponto_virgula >
Token Consumido(10): < identifier, k >
Token Consumido(10): < assign >
Token Consumido(10): < abre_parent >
Token Consumido(10): < abre_parent >
Token Consumido(10): < identifier, i >
Token Consumido(10): < mult_mulop >
Token Consumido(10): < num, 5 >
Token Consumido(10): < fecha_parent >
Token Consumido(10): < menos_addop >
Token Consumido(10): < abre_parent >
Token Consumido(10): < identifier, i >
Token Consumido(10): < mult_mulop >
Token Consumido(10): < num, 50 >
Token Consumido(10): < fecha_parent >
Token Consumido(10): < fecha_parent >
Token Consumido(10): < div_mulop >
Token Consumido(10): < num, 10 >
Token Consumido(10): < ponto_virgula >
Token Consumido(11): < identifier, j >
Token Consumido(11): < assign >
Token Consumido(11): < identifier, i >
Token Consumido(11): < mult_mulop >
Token Consumido(11): < num, 10 >
Token Consumido(11): < ponto_virgula >
Token Consumido(12): < identifier, k >
Token Consumido(12): < assign >
Token Consumido(12): < abre_parent >
Token Consumido(12): < identifier, i >
Token Consumido(12): < mult_mulop >
Token Consumido(12): < identifier, j >
Token Consumido(12): < fecha_parent >
Token Consumido(12): < div_mulop >
Token Consumido(12): < identifier, k >
Token Consumido(12): < ponto_virgula >
Token Consumido(13): < identifier, k >
Token Consumido(13): < assign >
Token Consumido(13): < num, 4 >
Token Consumido(13): < soma_addop >
Token Consumido(13): < identifier, a >
Token Consumido(13): < ponto_virgula >
Token Consumido(14): < write >
Token Consumido(14): < abre_parent >
Token Consumido(14): < identifier, nome >
Token Consumido(14): < fecha_parent >
Token Consumido(14): < ponto_virgula >
Token Consumido(15): < write >
Token Consumido(15): < abre_parent >
Token Consumido(15): < literal, ", os n?ºmeros gerados sao: " >
Token Consumido(15): < fecha_parent >
Token Consumido(15): < ponto_virgula >
Token Consumido(16): < write >
Token Consumido(16): < abre_parent >
Token Consumido(16): < identifier, i >
Token Consumido(16): < fecha_parent >
Token Consumido(16): < ponto_virgula >
Token Consumido(17): < write >
Token Consumido(17): < abre_parent >
Token Consumido(17): < identifier, j >
Token Consumido(17): < fecha_parent >
Token Consumido(17): < ponto_virgula >
Token Consumido(18): < write >
Token Consumido(18): < abre_parent >
Token Consumido(18): < identifier, k >
Token Consumido(18): < fecha_parent >
Token Consumido(18): < ponto_virgula >


\end{lstlisting}\newline

% Teste 6


%-------------------Teste 6 ------------------------
\section{Teste 6}


\begin{lstlisting}[caption={Teste6.txt},label={lst:label},language=C]
init
a, b, c, maior is integer;
write("Digite uma idade: ");
read(a);
write("Digite outra idade: ");
read(b);
write("Digite mais uma idade: ");
read(c;
maior := 0;
if ( a>b and a>c )
maior := a;
else
if (b>c)
maior := b;
else
maior := c;
write("Maior idade: ");
write(maior);
end
\end{lstlisting}

Para o codigo acima o compilador obteve a seguinte saida :

\begin{lstlisting}[caption={Saida para o Codigo de teste  : Teste6.txt},label={Entrada 1},language=C]
**** Tokens lidos ****
< init_program >
< identifier, a >
< virgula >
< identifier, b >
< virgula >
< identifier, c >
< virgula >
< identifier, maior >
< is_decl >
< integer_type >
< ponto_virgula >
< write >
< abre_parent >
< literal, "Digite uma idade: " >
< fecha_parent >
< ponto_virgula >
< read >
< abre_parent >
< identifier, a >
< fecha_parent >
< ponto_virgula >
< write >
< abre_parent >
< literal, "Digite outra idade: " >
< fecha_parent >
< ponto_virgula >
< read >
< abre_parent >
< identifier, b >
< fecha_parent >
< ponto_virgula >
< write >
< abre_parent >
< literal, "Digite mais uma idade: " >
< fecha_parent >
< ponto_virgula >
< read >
< abre_parent >
< identifier, c >
< ponto_virgula >
< identifier, maior >
< assign >
< num, 0 >
< ponto_virgula >
< if >
< abre_parent >
< identifier, a >
< greater_than_relop >
< identifier, b >
< and_mulop >
< identifier, a >
< greater_than_relop >
< identifier, c >
< fecha_parent >
< identifier, maior >
< assign >
< identifier, a >
< ponto_virgula >
< else >
< if >
< abre_parent >
< identifier, b >
< greater_than_relop >
< identifier, c >
< fecha_parent >
< identifier, maior >
< assign >
< identifier, b >
< ponto_virgula >
< else >
< identifier, maior >
< assign >
< identifier, c >
< ponto_virgula >
< write >
< abre_parent >
< literal, "Maior idade: " >
< fecha_parent >
< ponto_virgula >
< write >
< abre_parent >
< identifier, maior >
< fecha_parent >
< ponto_virgula >
< end >



**** Inicio Parser ****
Token Consumido(1): < init_program >
Token Consumido(2): < identifier, a >
Token Consumido(2): < virgula >
Token Consumido(2): < identifier, b >
Token Consumido(2): < virgula >
Token Consumido(2): < identifier, c >
Token Consumido(2): < virgula >
Token Consumido(2): < identifier, maior >
Token Consumido(2): < is_decl >
Token Consumido(2): < integer_type >
Token Consumido(2): < ponto_virgula >
Error in: identList; Error(3): Token n?£o esperado:< write >
Fim de arquivo inesperado.
\end{lstlisting}
No Código acima, não foi declarado o begin. Assim o comando write foi dado como inesperado, pois era esperado o identificador begin.


\subsection{Correção do Teste 6}
Foram feitas as devidas correções no código acima para adequar as regras da gramatica da linguagem .Assim, obtemos o seguinte resultado.


\begin{lstlisting}[caption={Saida Correta para o Codigo de teste  : Teste6.txt},label={Entrada 1},language=C]


**** Tokens lidos ****
< init_program >
< identifier, a >
< virgula >
< identifier, b >
< virgula >
< identifier, c >
< virgula >
< identifier, maior >
< is_decl >
< integer_type >
< ponto_virgula >
< begin >
< write >
< abre_parent >
< literal, "Digite uma idade: " >
< fecha_parent >
< ponto_virgula >
< read >
< abre_parent >
< identifier, a >
< fecha_parent >
< ponto_virgula >
< write >
< abre_parent >
< literal, "Digite outra idade: " >
< fecha_parent >
< ponto_virgula >
< read >
< abre_parent >
< identifier, b >
< fecha_parent >
< ponto_virgula >
< write >
< abre_parent >
< literal, "Digite mais uma idade: " >
< fecha_parent >
< ponto_virgula >
< read >
< abre_parent >
< identifier, c >
< fecha_parent >
< ponto_virgula >
< identifier, maior >
< assign >
< num, 0 >
< ponto_virgula >
< if >
< abre_parent >
< abre_parent >
< identifier, a >
< greater_than_relop >
< identifier, b >
< fecha_parent >
< and_mulop >
< abre_parent >
< identifier, a >
< greater_than_relop >
< identifier, c >
< fecha_parent >
< fecha_parent >
< begin >
< identifier, maior >
< assign >
< identifier, a >
< ponto_virgula >
< end >
< else >
< begin >
< if >
< abre_parent >
< identifier, b >
< greater_than_relop >
< identifier, c >
< fecha_parent >
< begin >
< identifier, maior >
< assign >
< identifier, b >
< ponto_virgula >
< end >
< else >
< begin >
< identifier, maior >
< assign >
< identifier, c >
< ponto_virgula >
< end >
< end >
< write >
< abre_parent >
< literal, "Maior idade: " >
< fecha_parent >
< ponto_virgula >
< write >
< abre_parent >
< identifier, maior >
< fecha_parent >
< ponto_virgula >
< stop_program >



**** Inicio Parser ****
Token Consumido(1): < init_program >
Token Consumido(2): < identifier, a >
Token Consumido(2): < virgula >
Token Consumido(2): < identifier, b >
Token Consumido(2): < virgula >
Token Consumido(2): < identifier, c >
Token Consumido(2): < virgula >
Token Consumido(2): < identifier, maior >
Token Consumido(2): < is_decl >
Token Consumido(2): < integer_type >
Token Consumido(2): < ponto_virgula >
Token Consumido(3): < begin >
Token Consumido(4): < write >
Token Consumido(4): < abre_parent >
Token Consumido(4): < literal, "Digite uma idade: " >
Token Consumido(4): < fecha_parent >
Token Consumido(4): < ponto_virgula >
Token Consumido(5): < read >
Token Consumido(5): < abre_parent >
Token Consumido(5): < identifier, a >
Token Consumido(5): < fecha_parent >
Token Consumido(5): < ponto_virgula >
Token Consumido(6): < write >
Token Consumido(6): < abre_parent >
Token Consumido(6): < literal, "Digite outra idade: " >
Token Consumido(6): < fecha_parent >
Token Consumido(6): < ponto_virgula >
Token Consumido(7): < read >
Token Consumido(7): < abre_parent >
Token Consumido(7): < identifier, b >
Token Consumido(7): < fecha_parent >
Token Consumido(7): < ponto_virgula >
Token Consumido(8): < write >
Token Consumido(8): < abre_parent >
Token Consumido(8): < literal, "Digite mais uma idade: " >
Token Consumido(8): < fecha_parent >
Token Consumido(8): < ponto_virgula >
Token Consumido(9): < read >
Token Consumido(9): < abre_parent >
Token Consumido(9): < identifier, c >
Token Consumido(9): < fecha_parent >
Token Consumido(9): < ponto_virgula >
Token Consumido(10): < identifier, maior >
Token Consumido(10): < assign >
Token Consumido(10): < num, 0 >
Token Consumido(10): < ponto_virgula >
Token Consumido(11): < if >
Token Consumido(11): < abre_parent >
Token Consumido(11): < abre_parent >
Token Consumido(11): < identifier, a >
Token Consumido(11): < greater_than_relop >
Token Consumido(11): < identifier, b >
Token Consumido(11): < fecha_parent >
Token Consumido(11): < and_mulop >
Token Consumido(11): < abre_parent >
Token Consumido(11): < identifier, a >
Token Consumido(11): < greater_than_relop >
Token Consumido(11): < identifier, c >
Token Consumido(11): < fecha_parent >
Token Consumido(11): < fecha_parent >
Token Consumido(11): < begin >
Token Consumido(12): < identifier, maior >
Token Consumido(12): < assign >
Token Consumido(12): < identifier, a >
Token Consumido(12): < ponto_virgula >
Token Consumido(13): < end >
Token Consumido(13): < else >
Token Consumido(13): < begin >
Token Consumido(14): < if >
Token Consumido(14): < abre_parent >
Token Consumido(14): < identifier, b >
Token Consumido(14): < greater_than_relop >
Token Consumido(14): < identifier, c >
Token Consumido(14): < fecha_parent >
Token Consumido(14): < begin >
Token Consumido(15): < identifier, maior >
Token Consumido(15): < assign >
Token Consumido(15): < identifier, b >
Token Consumido(15): < ponto_virgula >
Token Consumido(16): < end >
Token Consumido(16): < else >
Token Consumido(16): < begin >
Token Consumido(17): < identifier, maior >
Token Consumido(17): < assign >
Token Consumido(17): < identifier, c >
Token Consumido(17): < ponto_virgula >
Token Consumido(18): < end >
Token Consumido(19): < end >
Token Consumido(20): < write >
Token Consumido(20): < abre_parent >
Token Consumido(20): < literal, "Maior idade: " >
Token Consumido(20): < fecha_parent >
Token Consumido(20): < ponto_virgula >
Token Consumido(21): < write >
Token Consumido(21): < abre_parent >
Token Consumido(21): < identifier, maior >
Token Consumido(21): < fecha_parent >
Token Consumido(21): < ponto_virgula >


\end{lstlisting}\newline

% Teste 7



%-------------------Teste 7 ------------------------
\section{Teste 7}



\begin{lstlisting}[caption={Teste7.txt},label={lst:label},language=C]
% Programa de Teste Calculo de idade %
init
	cont_ is integer;
	media, idade, soma_ is integer;
begin
	cont_ := 5;
	soma := 0;
	do
		write(“Altura:” );
		read (altura);
		soma := soma + altura;
		cont_ := cont_ - 1;
	while(cont_ > 0);
	write(“Media: ”);
	write (soma / qtd);
stop
\end{lstlisting}

Para o codigo acima o compilador obteve a seguinte saida :

\begin{lstlisting}[caption={Saida para o Codigo de teste  : Teste.txt},label={Entrada 1},language=C]
**** Tokens lidos ****
< init_program >
< identifier, cont_ >
< is_decl >
< integer_type >
< ponto_virgula >
< identifier, media >
< virgula >
< identifier, idade >
< virgula >
< identifier, soma_ >
< is_decl >
< integer_type >
< ponto_virgula >
< begin >
< identifier, cont_ >
< assign >
< num, 5 >
< ponto_virgula >
< identifier, soma >
< assign >
< num, 0 >
< ponto_virgula >
< do >
< write >
< abre_parent >
Error  na Linha(9): Token 'â' Nao esperado
Error  na Linha(9): Token '?' Nao esperado
Error  na Linha(9): Token '?' Nao esperado
< identifier, Altura >
Error  na Linha(9): Token ':' Nao esperado
Error  na Linha(9): Token '?' Nao esperado
Error  na Linha(9): Token '?' Nao esperado
< fecha_parent >
< ponto_virgula >
< read >
< abre_parent >
< identifier, altura >
< fecha_parent >
< ponto_virgula >
< identifier, soma >
< assign >
< identifier, soma >
< soma_addop >
< identifier, altura >
< ponto_virgula >
< identifier, cont_ >
< assign >
< identifier, cont_ >
< menos_addop >
< num, 1 >
< ponto_virgula >
< while >
< abre_parent >
< identifier, cont_ >
< greater_than_relop >
< num, 0 >
< fecha_parent >
< ponto_virgula >
< write >
< abre_parent >
Error  na Linha(14): Token 'â' Nao esperado
Error  na Linha(14): Token '?' Nao esperado
Error  na Linha(14): Token '?' Nao esperado
< identifier, media >
Error  na Linha(14): Token ':' Nao esperado
Error  na Linha(14): Token 'â' Nao esperado
Error  na Linha(14): Token '?' Nao esperado
Error  na Linha(14): Token '?' Nao esperado
< fecha_parent >
< ponto_virgula >
< write >
< abre_parent >
< identifier, soma >
< div_mulop >
< identifier, qtd >
< fecha_parent >
< ponto_virgula >
< stop_program >



**** Inicio Parser ****
Token Consumido(2): < init_program >
Token Consumido(3): < identifier, cont_ >
Token Consumido(3): < is_decl >
Token Consumido(3): < integer_type >
Token Consumido(3): < ponto_virgula >
Token Consumido(4): < identifier, media >
Token Consumido(4): < virgula >
Token Consumido(4): < identifier, idade >
Token Consumido(4): < virgula >
Token Consumido(4): < identifier, soma_ >
Token Consumido(4): < is_decl >
Token Consumido(4): < integer_type >
Token Consumido(4): < ponto_virgula >
Token Consumido(5): < begin >
Token Consumido(6): < identifier, cont_ >
Token Consumido(6): < assign >
Token Consumido(6): < num, 5 >
Token Consumido(6): < ponto_virgula >
Token Consumido(7): < identifier, soma >
Token Consumido(7): < assign >
Token Consumido(7): < num, 0 >
Token Consumido(7): < ponto_virgula >
Token Consumido(8): < do >
Token Consumido(9): < write >
Token Consumido(9): < abre_parent >
Token Consumido(9): < identifier, Altura >
Token Consumido(9): < fecha_parent >
Token Consumido(9): < ponto_virgula >
Token Consumido(10): < read >
Token Consumido(10): < abre_parent >
Token Consumido(10): < identifier, altura >
Token Consumido(10): < fecha_parent >
Token Consumido(10): < ponto_virgula >
Token Consumido(11): < identifier, soma >
Token Consumido(11): < assign >
Token Consumido(11): < identifier, soma >
Token Consumido(11): < soma_addop >
Token Consumido(11): < identifier, altura >
Token Consumido(11): < ponto_virgula >
Token Consumido(12): < identifier, cont_ >
Token Consumido(12): < assign >
Token Consumido(12): < identifier, cont_ >
Token Consumido(12): < menos_addop >
Token Consumido(12): < num, 1 >
Token Consumido(12): < ponto_virgula >
Token Consumido(13): < while >
Token Consumido(13): < abre_parent >
Token Consumido(13): < identifier, cont_ >
Token Consumido(13): < greater_than_relop >
Token Consumido(13): < num, 0 >
Token Consumido(13): < fecha_parent >
Token Consumido(13): < ponto_virgula >
Token Consumido(14): < write >
Token Consumido(14): < abre_parent >
Token Consumido(14): < identifier, media >
Token Consumido(14): < fecha_parent >
Token Consumido(14): < ponto_virgula >
Token Consumido(15): < write >
Token Consumido(15): < abre_parent >
Token Consumido(15): < identifier, soma >
Token Consumido(15): < div_mulop >
Token Consumido(15): < identifier, qtd >
Token Consumido(15): < fecha_parent >
Token Consumido(15): < ponto_virgula >
\end{lstlisting}
O Código acima esta escrito conforme a gramatica da linguagem . Não foi reportado nenhum erro.


\newline


%-------------------Teste 8 ------------------------
\section{Teste 8}



\begin{lstlisting}[caption={Teste8.txt},label={lst:label},language=C]
init 
a, b is integer;
begin
if ( a > b) begin
write ("O maior numero e: ");
write (a);
end else begin
write ("O menor numero e: ");
write (b);
end
stop
\end{lstlisting}

Para o codigo acima o compilador obteve a seguinte saida :

\begin{lstlisting}[caption={Saida para o Codigo de teste  : Teste8.txt},label={Entrada 1},language=C]
**** Tokens lidos ****
< init_program >
< identifier, a >
< virgula >
< identifier, b >
< is_decl >
< integer_type >
< ponto_virgula >
< begin >
< if >
< abre_parent >
< identifier, a >
< greater_than_relop >
< identifier, b >
< fecha_parent >
< begin >
< write >
< abre_parent >
< literal, "O maior numero e: " >
< fecha_parent >
< ponto_virgula >
< write >
< abre_parent >
< identifier, a >
< fecha_parent >
< ponto_virgula >
< end >
< else >
< begin >
< write >
< abre_parent >
< literal, "O menor numero e: " >
< fecha_parent >
< ponto_virgula >
< write >
< abre_parent >
< identifier, b >
< fecha_parent >
< ponto_virgula >
< end >
< stop_program >



**** Inicio Parser ****
Token Consumido(1): < init_program >
Token Consumido(2): < identifier, a >
Token Consumido(2): < virgula >
Token Consumido(2): < identifier, b >
Token Consumido(2): < is_decl >
Token Consumido(2): < integer_type >
Token Consumido(2): < ponto_virgula >
Token Consumido(3): < begin >
Token Consumido(4): < if >
Token Consumido(4): < abre_parent >
Token Consumido(4): < identifier, a >
Token Consumido(4): < greater_than_relop >
Token Consumido(4): < identifier, b >
Token Consumido(4): < fecha_parent >
Token Consumido(4): < begin >
Token Consumido(5): < write >
Token Consumido(5): < abre_parent >
Token Consumido(5): < literal, "O maior numero e: " >
Token Consumido(5): < fecha_parent >
Token Consumido(5): < ponto_virgula >
Token Consumido(6): < write >
Token Consumido(6): < abre_parent >
Token Consumido(6): < identifier, a >
Token Consumido(6): < fecha_parent >
Token Consumido(6): < ponto_virgula >
Token Consumido(7): < end >
Token Consumido(7): < else >
Token Consumido(7): < begin >
Token Consumido(8): < write >
Token Consumido(8): < abre_parent >
Token Consumido(8): < literal, "O menor numero e: " >
Token Consumido(8): < fecha_parent >
Token Consumido(8): < ponto_virgula >
Token Consumido(9): < write >
Token Consumido(9): < abre_parent >
Token Consumido(9): < identifier, b >
Token Consumido(9): < fecha_parent >
Token Consumido(9): < ponto_virgula >
Token Consumido(10): < end >

\end{lstlisting}O Código acima foi criado seguindo as regras de derivação da gramatica da linguagem. Não foram encontrados erros.