\chapter{Introdução}

Um compilador é um programa de computador (ou um grupo de programas) que, a partir de um código fonte escrito em uma linguagem compilada, cria um programa semanticamente equivalente, porém escrito em outra linguagem, código objeto. Classicamente, um compilador traduz um programa de uma linguagem textual facilmente entendida por um ser humano para uma linguagem de máquina , específica para um processador e sistema operacional. Atualmente, porém, são comuns compiladores que geram código para uma máquina virtual que é, depois, interpretada por um interpretador. Ele é chamado compilador por razões históricas; nos primeiros anos da programação automática, existiam programas que percorriam bibliotecas de sub-rotinas e as reunia, ou compilava, as sub-rotinas necessárias para executar uma determinada tarefa.

O nome "compilador" é usado principalmente para os programas que traduzem o código fonte de uma linguagem de programação de alto nível para uma linguagem de programação de baixo nível (por exemplo, Assembly ou código de máquina). Contudo alguns autores citam exemplos de compiladores que traduzem para linguagens de alto nível como C. Para alguns autores um programa que faz uma tradução entre linguagens de alto nível é normalmente chamado um tradutor, filtro ou conversor de linguagem. Um programa que traduz uma linguagem de programação de baixo nível para uma linguagem de programação de alto nível é um descompilador. Um programa que faz uma tradução entre uma linguagem de montagem e o código de máquina é denominado montador (assembler). Um programa que faz uma tradução entre o código de máquina e uma linguagem de montagem é denominado desmontador (disassembler). Se o programa compilado pode ser executado em um computador cuja CPU ou sistema operacional é diferente daquele em que o compilador é executado, o compilador é conhecido como um compilador cruzado.