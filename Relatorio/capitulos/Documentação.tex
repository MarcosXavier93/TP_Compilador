\chapter{Documentação}
O Analisador Léxico é o processo de analisar a entrada de linhas de caracteres (tal como o código-fonte de um programa de computador) e produzir uma sequência de símbolos chamado "símbolos léxicos" (lexical tokens), ou somente "símbolos" (tokens), que podem ser manipulados mais facilmente por um parser (leitor de saída). Para tal procedimento, foi necessário a realização a partir do modelo de uma maquina de estados, processando os caracteres do arquivo fonte, montando lexemas e vinculando a um token especifico, verificando também a Tabela de Símbolos e respeitando a gramatica da linguagem.
O Projeto foi desenvolvido em conjunto com as plataformas Visual Studio Code e Apache Netbeans, a principio todo o código seria criado somente "a mão" criando as pastas e arquivos manualmente, porém, devido a algumas facilidades que o Netbeans nos oferece, o projeto foi alterado em alguns parâmetros para que seja possível abrir no Netbeans.
Uma dessas facilidades é a opção de geração da Documentação do projeto, assim o fizemos e esta documentação gerada automaticamente pelo Netbeans se encontra no diretório: " /TP{\_}Compilador/Projeto/dist/javadoc/" . Para esta implementação foi acatado o modelo abordado pelo autor do livro base da disciplina: Aho .