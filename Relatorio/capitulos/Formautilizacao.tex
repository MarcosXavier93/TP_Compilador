\chapter{Forma de Utilização}
\usepackage{}
Para que seja possivel a utilização deste programa, basta entrar no terminal e navegar até o diretório: "TP{\_}Compilador/src/", e então, digitar o seguinte comando :
\begin{lstlisting}[language=Java, caption={Entrada terminal},label={Terminal para inicio do Programa}]
    javac Main.java
    java Main
\end{lstlisting}
Acima podemos ver duas linhas de comando, a primeira serve para que possa compilar a classe Main.java e então, é criado o executável e assim, podemos acessar a classe a partir da linha seguinte, java Main .
\newline
Definimos qual arquivo de teste que o programa deve compilar dentro da classe Main, no seguinte trecho de código:
\begin{lstlisting}[language=Java, caption={Indicando arquivo para teste},label={Terminal para inicio do Programa}]
    public static void main(String[] args) {
		ArrayList<Token> tokens = new ArrayList<Token> ();
		Lexer L = null;
		int line = -5;
		try {
			L = new Lexer("codigos_teste/Teste2.txt");
			
\end{lstlisting}
\newline 
Na ultima linha do trecho do código destacado acima, mostra como determinamos qual programa nosso compilador irá testar.
Veja que criamos uma pasta chamada testes para que seja incluídos somente os arquivos de teste do programa. Foram criados 10 arquivos de teste.




