\chapter{Testes realizados}
Para testar todo o compilador construído, foram propostos sete exemplos de teste, abordando algumas possibilidade de programas a serem compilados.
Apartir desses arquivos podemos visualizar os seguintes retorno do compilador :
\newline



\begin{lstlisting}[caption={Teste1.txt},label={lst:label},language=Comsol]

init
a, b, $c, is integer;
result is real;
write("Digite o valor de a:"
read (a);
write("Digite o valor de c:"
read (c);
b := 10
result := (a * c)/(b + 5 - 345);
write("O resultado e:");
write(result);
stop


\end{lstlisting}\newline
Para o codigo acima  o compilador obeteve a seguinte saida :

\begin{lstlisting}[caption={Saida para o Codigo de teste  : Teste1.txt},label={Entrada 1},language=Comsol]
**** Tokens lidos ****
< identifier, a >
< virgula >
< virgula >
< identifier, b_1 >
< virgula >
< identifier, b_2 >
< virgula >
< identifier, cc >
< virgula >
< num, 10 >
Error(1): Token ' ' inválido
Error(1): Token ':' inválido
< integer_type >
< ponto_virgula >
< init_program >
< write >
< abre_parent >
< literal, "Entre com o valor de a:" >
< fecha_parent >
< ponto_virgula >
< read >
< abre_parent >
< identifier, a >
< fecha_parent >
< ponto_virgula >
< identifier, b_1 >
< assign >
< identifier, a >
< mult_mulop >
< identifier, a >
< ponto_virgula >
< write >
< abre_parent >
< literal, "O valor de b1 e:" >
< fecha_parent >
< ponto_virgula >
< write >
< abre_parent >
< identifier, b_1 >
< fecha_parent >
< ponto_virgula >
< identifier, b_2 >
< equal_relop >
< identifier, b >
< soma_addop >
< identifier, a >
< div_mulop >
< num, 2 >
< mult_mulop >
< abre_parent >
< identifier, a >
< soma_addop >
< num, 5 >
< fecha_parent >
< ponto_virgula >
< write >
< abre_parent >
< literal, "O valor de b1 e:" >
< fecha_parent >
< ponto_virgula >
< identifier, Write >
< abre_parent >
< identifier, b1 >
< fecha_parent >
< ponto_virgula >
< stop_program >



**** Tabela de símbolos ****
Entrada         |               Mais info
string
Write
not
end
is
b1
or
b_2
b_1
real
if
cc
while
do
stop
read
init
else
integer
write
begin
b
a
and


\end{lstlisting}\newline


\begin{lstlisting}[caption={Teste2.txt},label={lst:label},language=Comsol]
a, _valor, b_1, b_2, cc, 10_a : integer;
init
write("Entre com o valor de a:");
read (a);
b_1 := a * a;
write("O valor de b1 e:");
write (b_1);
b_2 = b + a/2 * (a + 5);
write("O valor de b1 e:");
Write (b1);
stop


\end{lstlisting}\newline

Para o codigo acima o compilador obeteve a seguinte saida :

\begin{lstlisting}[caption={Saida para o Codigo de teste  : Teste1.txt},label={Entrada 1},language=Comsol]
-------Tokens Reconhecidos pelo Compilador-------
< identifier, a >
< virgula >
< virgula >
< identifier, b_1 >
< virgula >
< identifier, b_2 >
< virgula >
< identifier, cc >
< virgula >
< num, 10 >
Error(1): Token ' ' inválido
Error(1): Token ':' inválido
< integer_type >
< ponto_virgula >
< init_program >
< write >
< abre_parent >
< literal, "Entre com o valor de a:" >
< fecha_parent >
< ponto_virgula >
< read >
< abre_parent >
< identifier, a >
< fecha_parent >
< ponto_virgula >
< identifier, b_1 >
< assign >
< identifier, a >
< mult_mulop >
< identifier, a >
< ponto_virgula >
< write >
< abre_parent >
< literal, "O valor de b1 e:" >
< fecha_parent >
< ponto_virgula >
< write >
< abre_parent >
< identifier, b_1 >
< fecha_parent >
< ponto_virgula >
< identifier, b_2 >
< equal_relop >
< identifier, b >
< soma_addop >
< identifier, a >
< div_mulop >
< num, 2 >
< mult_mulop >
< abre_parent >
< identifier, a >
< soma_addop >
< num, 5 >
< fecha_parent >
< ponto_virgula >
< write >
< abre_parent >
< literal, "O valor de b1 e:" >
< fecha_parent >
< ponto_virgula >
< identifier, Write >
< abre_parent >
< identifier, b1 >
< fecha_parent >
< ponto_virgula >
< stop_program >



------------- Imprimindo a Tabela de símbolos-------------
                Entrada
string
Write
not
end
is
b1
or
b_2
b_1
real
if
cc
while
do
stop
read
init
else
integer
write
begin
b
a
and

\end{lstlisting}\newline

\begin{lstlisting}[caption={Teste3.txt},label={lst:label},language=Comsol]
% Programa de Teste
Calculo de idade%
INIT
cont is int;
media, idade, soma is integer;
begin
cont := 5;
soma := 0;
do
write("Altura:" );
read (altura);
soma := soma+altura;
cont := cont - 1;
while(cont > 0)
media := soma / qts
write("Media: ");
write (media);
STOP
\end{lstlisting}


Para o codigo acima o compilador obeteve a seguinte saida :

\begin{lstlisting}[caption={Saida para o Codigo de teste  : Teste3.txt},label={Entrada 1},language=Comsol]

-------Tokens Reconhecidos pelo Compilador-------
< identifier, INIT >
< identifier, cont >
< is_decl >
< identifier, int >
< ponto_virgula >
< identifier, media >
< virgula >
< identifier, idade >
< virgula >
< identifier, soma >
< is_decl >
< integer_type >
< ponto_virgula >
< begin >
< identifier, cont >
< assign >
< num, 5 >
< ponto_virgula >
< identifier, soma >
< assign >
< num, 0 >
< ponto_virgula >
< do >
< write >
< abre_parent >
< literal, "Altura:" >
< fecha_parent >
< ponto_virgula >
< read >
< abre_parent >
< identifier, altura >
< fecha_parent >
< ponto_virgula >
< identifier, soma >
< assign >
< identifier, soma >
< soma_addop >
< identifier, altura >
< ponto_virgula >
< identifier, cont >
< assign >
< identifier, cont >
< menos_addop >
< num, 1 >
< ponto_virgula >
< while >
< abre_parent >
< identifier, cont >
< greater_than_relop >
< num, 0 >
< fecha_parent >
< identifier, media >
< assign >
< identifier, soma >
< div_mulop >
< identifier, qts >
< write >
< abre_parent >
< literal, "Media: " >
< fecha_parent >
< ponto_virgula >
< write >
< abre_parent >
< identifier, media >
< fecha_parent >
< ponto_virgula >
< identifier, STOP >



------------- Imprimindo a Tabela de símbolos-------------
                Entrada
STOP
string
int
INIT
cont
not
end
altura
idade
media
is
or
qts
soma
real
if
while
do
stop
read
init
else
integer
write
begin
and
\end{lstlisting}


\begin{lstlisting}[caption={Teste4.txt},label={lst:label},language=Comsol]
init
% Outro programa de teste
i, j, k, @total is integer;
nome is string
write("Digite o seu nome: ");
read(nome);
write("Digite um valor inteiro: );
read (I);
k := i * (5-i * 50 / 10;
j := i * 10;
k := i* j / k;
k := 4 + a $;
write(nome);
write(", os números gerados sao: ");
write(i);
write(j);
write(k);

\end{lstlisting}

Para o codigo acima o compilador obeteve a seguinte saida :

\begin{lstlisting}[caption={Saida para o Codigo de teste  : Teste4.txt},label={Entrada 1},language=Comsol]
-------Tokens Reconhecidos pelo Compilador-------
< init_program >
< mult_mulop >
< abre_parent >
< num, 5 >
< menos_addop >
< identifier, i >
< mult_mulop >
< num, 50 >
< div_mulop >
< num, 10 >
< ponto_virgula >
< identifier, j >
< assign >
< identifier, i >
< mult_mulop >
< num, 10 >
< ponto_virgula >
< identifier, k >
< assign >
< identifier, i >
< mult_mulop >
< identifier, j >
< div_mulop >
< identifier, k >
< ponto_virgula >
< identifier, k >
< assign >
< num, 4 >
< soma_addop >
< identifier, a >
Error(12): Token '$' inválido
< ponto_virgula >
< write >
< abre_parent >
< identifier, nome >
< fecha_parent >
< ponto_virgula >
< write >
< abre_parent >
< literal, ", os números gerados sao: " >
< fecha_parent >
< ponto_virgula >
< write >
< abre_parent >
< identifier, i >
< fecha_parent >
< ponto_virgula >
< write >
< abre_parent >
< identifier, j >
< fecha_parent >
< ponto_virgula >
< write >
< abre_parent >
< identifier, k >
< fecha_parent >
< ponto_virgula >



------------- Imprimindo a Tabela de símbolos-------------
                Entrada
string
not
end
is
or
real
if
while
do
stop
k
nome
j
read
i
init
else
integer
write
begin
a
and

\end{lstlisting}

\begin{lstlisting}[caption={Teste6.txt},label={lst:label},language=Comsol]
init
a, b, c, maior is integer;
write("Digite uma idade: ");
read(a);
write("Digite outra idade: ");
read(b);
write("Digite mais uma idade: ");
read(c;
maior := 0;
if ( a>b and a>c )
maior := a;
else
if (b>c)
maior := b;
else
maior := c;
write("Maior idade: ");
write(maior);
end

\end{lstlisting}

Para o codigo acima o compilador obeteve a seguinte saida :

\begin{lstlisting}[caption={Saida para o Codigo de teste  : Teste6.txt},label={Entrada 1},language=Comsol]
-------Tokens Reconhecidos pelo Compilador-------
< init_program >
< identifier, a >
< virgula >
< identifier, b >
< virgula >
< identifier, c >
< virgula >
< identifier, maior >
< is_decl >
< integer_type >
< ponto_virgula >
< write >
< abre_parent >
< literal, "Digite uma idade: " >
< fecha_parent >
< ponto_virgula >
< read >
< abre_parent >
< identifier, a >
< fecha_parent >
< ponto_virgula >
< write >
< abre_parent >
< literal, "Digite outra idade: " >
< fecha_parent >
< ponto_virgula >
< read >
< abre_parent >
< identifier, b >
< fecha_parent >
< ponto_virgula >
< write >
< abre_parent >
< literal, "Digite mais uma idade: " >
< fecha_parent >
< ponto_virgula >
< read >
< abre_parent >
< identifier, c >
< ponto_virgula >
< identifier, maior >
< assign >
< num, 0 >
< ponto_virgula >
< if >
< abre_parent >
< identifier, a >
< greater_than_relop >
< identifier, b >
< and_mulop >
< identifier, a >
< greater_than_relop >
< identifier, c >
< fecha_parent >
< identifier, maior >
< assign >
< identifier, a >
< ponto_virgula >
< else >
< if >
< abre_parent >
< identifier, b >
< greater_than_relop >
< identifier, c >
< fecha_parent >
< identifier, maior >
< assign >
< identifier, b >
< ponto_virgula >
< else >
< identifier, maior >
< assign >
< identifier, c >
< ponto_virgula >
< write >
< abre_parent >
< literal, "Maior idade: " >
< fecha_parent >
< ponto_virgula >
< write >
< abre_parent >
< identifier, maior >
< fecha_parent >
< ponto_virgula >
< end >



------------- Imprimindo a Tabela de símbolos-------------
                Entrada
string
maior
not
end
is
or
real
if
while
do
stop
read
init
else
integer
write
c
begin
b
a
and

\end{lstlisting}
\begin{lstlisting}[caption={Teste7.txt},label={lst:label},language=Comsol]
% Programa de Teste%
Calculo de idade
init
	cont_ is int;
	media, idade, soma_ is integer;
begin
	cont_ = 5;
	soma = 0;

	do
		write(“Altura:” );
		read (altura);
		soma := soma altura;
		cont_ := cont_ - 1;
	while(cont_ > 0)

	write(“Media: ”);
	write (soma / qtd);

stop
\end{lstlisting}

Para o codigo acima o compilador obeteve a seguinte saida :

\begin{lstlisting}[caption={Saida para o Codigo de teste  : Teste7.txt},label={Entrada 1},language=Comsol]
-------Tokens Reconhecidos pelo Compilador-------
< identifier, Calculo >
< identifier, de >
< identifier, idade >
< init_program >
< identifier, cont_ >
< is_decl >
< identifier, int >
< ponto_virgula >
< identifier, media >
< virgula >
< identifier, idade >
< virgula >
< identifier, soma_ >
< is_decl >
< integer_type >
< ponto_virgula >
< begin >
< identifier, cont_ >
< equal_relop >
< num, 5 >
< ponto_virgula >
< identifier, soma >
< equal_relop >
< num, 0 >
< ponto_virgula >
< do >
< write >
< abre_parent >
Error(11): Token '“' inválido
< identifier, Altura >
Error(11): Token ':' inválido
< fecha_parent >
< ponto_virgula >
< read >
< abre_parent >
< identifier, altura >
< fecha_parent >
< ponto_virgula >
< identifier, soma >
< assign >
< identifier, soma >
< identifier, altura >
< ponto_virgula >
< identifier, cont_ >
< assign >
< identifier, cont_ >
< menos_addop >
< num, 1 >
< ponto_virgula >
< while >
< abre_parent >
< identifier, cont_ >
< greater_than_relop >
< num, 0 >
< fecha_parent >
< write >
< abre_parent >
Error(17): Token '“' inválido
< identifier, Media >
Error(17): Token ':' inválido
Error(17): Token '”' inválido
< fecha_parent >
< ponto_virgula >
< write >
< abre_parent >
< identifier, soma >
< div_mulop >
< identifier, qtd >
< fecha_parent >
< ponto_virgula >
< stop_program >



------------- Imprimindo a Tabela de símbolos-------------
                Entrada
string
int
not
end
altura
idade
media
is
or
Media
soma
real
if
while
Altura
do
stop
read
init
soma_
else
integer
write
qtd
de
begin
cont_
and
Calculo

\end{lstlisting}
\begin{lstlisting}[caption={Teste8.txt},label={lst:label},language=Comsol]
init
	a, b is integer;
	if( a >= b)
	read(a);
	end;
stop
\end{lstlisting}

Para o codigo acima o compilador obeteve a seguinte saida :

\begin{lstlisting}[caption={Saida para o Codigo de teste  : Teste8.txt},label={Entrada 1},language=Comsol]
-------Tokens Reconhecidos pelo Compilador-------
< init_program >
< identifier, a >
< virgula >
< identifier, b >
< is_decl >
< integer_type >
< ponto_virgula >
< if >
< abre_parent >
< identifier, a >
< greater_equals_relop >
< identifier, b >
< fecha_parent >
< read >
< abre_parent >
< identifier, a >
< fecha_parent >
< ponto_virgula >
< end >
< ponto_virgula >
< stop_program >



------------- Imprimindo a Tabela de símbolos-------------
                Entrada
string
not
end
is
or
real
if
while
do
stop
read
init
else
integer
write
begin
b
a
and

\end{lstlisting}
\begin{lstlisting}[caption={Teste9.txt},label={lst:label},language=Comsol]
init
	read(a);
	if(a == b) begin
		read(a);
	end;
stop


\end{lstlisting}

Para o codigo acima o compilador obeteve a seguinte saida :

\begin{lstlisting}[caption={Saida para o Codigo de teste  : Teste9.txt},label={Entrada 1},language=Comsol]

-------Tokens Reconhecidos pelo Compilador-------
< init_program >
< read >
< abre_parent >
< identifier, a >
< fecha_parent >
< ponto_virgula >
< if >
< abre_parent >
< identifier, a >
< equal_relop >
< equal_relop >
< identifier, b >
< fecha_parent >
< begin >
< read >
< abre_parent >
< identifier, a >
< fecha_parent >
< ponto_virgula >
< end >
< ponto_virgula >
< stop_program >



------------- Imprimindo a Tabela de símbolos-------------
                Entrada
string
not
end
is
or
real
if
while
do
stop
read
init
else
integer
write
begin
b
a
and
\end{lstlisting}
\begin{lstlisting}[caption={Teste10.txt},label={lst:label},language=Comsol]
init
	a;
	%Declarando errado  avariavel a%
	
	read(a);
	
stop
\end{lstlisting}

Para o codigo acima o compilador obeteve a seguinte saida :

\begin{lstlisting}[caption={Saida para o Codigo de teste  : Teste10.txt},label={Entrada 1},language=Comsol]

-------Tokens Reconhecidos pelo Compilador-------
< init_program >
< identifier, a >
< ponto_virgula >
< read >
< abre_parent >
< identifier, a >
< fecha_parent >
< ponto_virgula >
< stop_program >



------------- Imprimindo a Tabela de símbolos-------------
                Entrada
string
not
end
is
or
real
if
while
do
stop
read
init
else
integer
write
begin
a
and
              
\end{lstlisting}



\newline
\newline
