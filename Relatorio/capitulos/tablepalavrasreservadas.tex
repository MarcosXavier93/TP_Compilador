\begin{table}[H]
        \vspace{1.0em}
        \centering%
        \caption{Relação das Palavras Reservadas da Linguagem}%
        \begin{tabular} {p{5cm}p{11cm}}% Aqui é possível definir o tamanho de cada coluna. 
            \hline
            \textbf{Palavra Reservada}  & \textbf{Significado}\\
            \hline
            {init}             &   {Inicio do Programa}   \\
            \hline
            {stop}&{Indica a finalização do programa }\\
            \hline
            {is} & {Usado para determinador o tipo de uma variável} \\
            \hline
            

{integer}&{Tipo Inteiro para variável} \\
              \hline
{string}&{Determina a variável como tipo string}\\
  \hline
{real} & {Determina a variável como tipo Real} \\
\hline


{if}&{Determina o inicio de um bloco de código sob uma condicional}\\
\hline

{begin}&{Deternia o inicio de uma condição do if}\\
\hline
{end}&{Determina o fim da condição do if}\\
\hline
{else}&{Determina a contraposição da condição do if}\\
\hline
{end else}&{Indica o Fim da condição do Else if}\\
\hline
{do}&{Determina a entrada de um laço }\\
\hline
{while}&{determina a condição para o laço Do}\\
\hline
{read}&{Prepara para ler e reservar a próxima variável a ser lida}\\
\hline
{write}&{Determina que ira imprimir na tela a literal que virá a seguir}\\
\hline
{not}&{Determina a negação do valor booleano de uma expressão}\\
\hline
{or}&{Determina a soma binaria entre dois valores }\\
\hline
{and}&{Determina a multiplicação binaria entre dois valores}\\

            \hline
        \end{tabular}
        \\\hspace{\linewidth}%
        \textbf{Fonte:} Documentação Fornecida pela Orientadora o qual pode ser consultado no diretório do projeto%
        \label{table:cronograma}
        \vspace{1.0em}
    \end{table}